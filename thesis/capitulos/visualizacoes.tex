\chapter{InterSCity Plot}
\label{cap:visualizacao}

Esse capítulo introduz as técnicas de visualização utilizadas no desenvolvimento
de uma ferramenta para a análise do tráfego de veículos, como foram selecionadas e desafios envolvidos.

\section{Modelo de dados}

Esta seção explica de onde vem os dados e quais suas características em detalhes
para um melhor entendimento e clareza do leitor, citando os trabalhos publicados sobre o InterSCimulator.
É importante entender os dados para que não haja confusão com a agregação dos dados, etc. Será
explicado todos os atributos da simulação e algumas métricas importantes como número
de viagens totais, etc.

\section{Seleção das Técnicas}

Aqui se explica de maneira explícita a seleção do trabalho de Alex Telea e as técnicas
apontadas por ele para análise do tráfego aéreo. Apontamos as similaridades entre
os dois contextos e as peculiaridades. Defenderemos também que mesmo se baseando
no trabalho, o mesmo não se encontra disponível e não é reproduzível. Pontuamos também
que a escolha das visualizações do seu trabalho respondem questões importantes
para o estudo do tráfego e que outras visualizações apresentadas em outros estudos citados
nos trabalhos relacionados podem ser adicionadas no futuro.

\section{Desafios Técnicos}

Nesta seção será descrito alguns desafios técnicos relevantes para o desenvolvimento da solução
que irá permitir a análise dos dados, como a seleção de ferramentas,
desafios de escalabilidade computacional e o uso de bibliotecas livres disponíveis.

\section{Resultados Preliminares}

Aqui são apresentados os experimentos com algumas visualizações iniciais já construídas
utilizando uma parte dos dados, descrevendo todo o pipeline envolvido, desde a seleção, à filtragem dos dados,
transformações executadas e por fim a visualização com a ferramenta escolhida.
Os resultados preliminares incluem uma animação da movimentação dos veículos e um mapa de densidade
como forma de validação de todo o pipeline e processamento dos dados.

