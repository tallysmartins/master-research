\chapter{Visualização do Tráfego de Veículos}
\label{cap:visualizacao}

Nesse capítulo eu introduzo a proposta de solução, com as técnicas de
visualização a serem utilizadas utilizadas no desenvolvimento de uma ferramenta
para a análise do tráfego de veículos, os desafios envolvidos e resultados
preliminares.

\section{Conjunto de dados}

Esta seção explica de onde vem os dados e quais suas características (tamanho,
organização e atributos), além de outros dados que serão derivados a partir dos
dados brutos iniciais. É importante entender os dados, como eles serão
agregados, etc.

\section{Seleção das Técnicas}

Aqui eu explico de maneira explícita a seleção do trabalho de Alex Telea e as
técnicas apontadas por ele para análise do tráfego aéreo. Apontamos as
similaridades entre os dois contextos e as peculiaridades. Pontuamos também que
a escolha das visualizações do seu trabalho respondem questões importantes para
o estudo do tráfego e que outras visualizações apresentadas em outros estudos
citados nos trabalhos relacionados podem ser adicionadas no futuro.

\section{Desafios Técnicos}

Aqui eu falo sucintamente sobre alguns desafios técnicos relevantes para o
desenvolvimento da solução que irá permitir a análise dos dados, como a seleção
de ferramentas, desafios de escalabilidade computacional e o uso de bibliotecas
livres disponíveis.

\section{Resultados Preliminares}

Resultados alcançados até o momento.

