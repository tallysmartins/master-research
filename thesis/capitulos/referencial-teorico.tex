\chapter{Referencial Teórico}
\label{cap:referencial-teorico}

Nesse capítulo eu apresento conceitos importantes sobre (i) dados de
movimentação, (ii) fontes de dados de movimentação a serem utilizadas, (iii) a
técnicas de bundling 

\section{Dados de Movimentação}

  Aqui eu caracterizo com mais detalhes dados de movimentação, seus atributos de
maneira geral e sua modelagem e relação com grafos dinâmicos, e esses acabam
sendo um subconjunto dessa área. Explico que podem ser dados
point-to-point e region-to-point e que nesse trabalho estamos interessados em
point-to-point com alta granularidade, ou seja, quando temos todos os pontos do
trajeto e não apenas origem destino. Abordo também um pouco da escalabilidade, pois
esse tipo de datasets costumam chegar facilmente a milhares de pontos.

\section{Fonte de dados}
  Nesta seção eu apresento as fontes de dados as quais pretende-se analisar a
aplicação das técnicas de visualização, no caso o InterSCimulator. 
  Aqui eu falo o que é o simulador e como ele é empregado para a obtenção dos
dados que compõe o rastro dos veículos e qual a estrutura dos dados gerados.
tamanho do dataset, atributos, etc. Ressalto também sua capacidade de gerar simulações multi-modais e como elas são
geradas a partir da pesquisa origem-destino.

\section{Visualização de Trajetórias}
  Apresento o bundling como técnica de simplificação da visualização  para exploração
de dados geoespaciais.

\subsection{Bundling}
    Aqui eu argumento sobre as vantagens de se utilizar bundling, pra que ele serve,
sua complexidade, desvantagens, variações do algoritmo, etc.

