%% ------------------------------------------------------------------------- %%
\chapter{Introdução}
\label{cap:introducao}

% Contexto e motivação
Aqui eu darei uma introdução do contexto e da motivação, argumentando sobre o aumento da geração de dados de
movimentação (geolocalizados) e suas aplicações em diversas áreas como carros,
pessoas,etc. Em seguida irei deixar claro o foco do trabalho no tráfego de veículos 
na cidade. Defendendo o porque de analisar esses dados na
perspectiva de estudos que buscam melhorias para o transporte urbano. Aponto
ainda outros estudos que usam dados de movimentação em outros contextos, como no tráfego de embarcações e aviões.

% Problema
Encadeado no parágrafo anterior eu apresento o problema em se fazer
uma análise da movimentação de veículos em uma grande cidade, considerando-se
os múltiplos modais existentes, como carros, ônibus, metrô, bicicletas. Levanto
que a visualização desses dados possui desafios de escalabilidade visual e computacional,

% A Proposta e justificativa
Apresento então o bundling como técnica para simplificar a visualização desse tipo de dado
e que ele pode ser usado na perspectiva dos diferentes modais de transporte.
Afirmo ainda que uma visualização como essa pode auxiliar à tomada
de decisões e na busca de melhorias na gestão da mobilidade urbana.

Por fim eu apresento a proposta do trabalho em se
desenvolver uma ferramenta para visualização interativa com a aplicação do bundling 
permitam enxergar uma grande quantidade de dados do trânsito
obtidos a partir de um simulador de cidades inteligentes. Defendo também que tal
ferramenta pode ainda ser utilizada de maneira a analisar quaisquer outras fontes
de dados de movimentação com características similares, como dados
reais de ônibus, táxi ou outros veículos que se movem nas ruas da cidades.

\section{Objetivos}
\label{sec:objetivos}

O principal objetivo do trabalho é desenvolver uma visualização que utilize o bundling
para visualizar a dinâmica da movimentação do tráfego na cidade considerando
os diferentes modais de transporte. Dado esse objetivo foram levantadas duas questões de pesquisa:

\begin{description}
  \item[QP1] Como o bundling pode auxiliar na identificação de padrões de movimentação nos diferentes modais de transporte?

  \item[QP2] Como o bundling pode auxiliar na visualização da simulação do tráfego de veículos?

\end{description}


%Para atingir o objetivo geral, os seguintes objetivos específicos devem ser atingidos:
%
%\begin{itemize}
%  \item Obtenção dos dados simulados do tráfego de veículos na cidade de São Paulo
%  \item Seleção de Técnicas de Visualização
%  \item Pesquisa de ferramentas e tecnologias para aplicação das técnicas
%  \item Implementação das técnicas de visualização sobre os dados da simulação
%  \item Análise e avaliação dos resultados
%\end{itemize}
