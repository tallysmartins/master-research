%% ------------------------------------------------------------------------- %%
\chapter{Introdução}
\label{cap:introducao}

% Contexto e motivação
Aqui eu darei uma introdução do contexto e da motivação, argumentando sobre o
aumento da geração de dados de movimentação (geolocalizados) e suas aplicações
para estudo das atividades de movimentação em diversas áreas como carros,
pessoas, etc. Em seguida irei deixar claro o foco do trabalho no tráfego de
veículos na cidade. Defendendo o porque de analisar esses dados na perspectiva
de estudos que buscam melhorias para o transporte urbano. 

% Problema
Encadeado no parágrafo anterior eu apresento o problema em se fazer uma análise
da movimentação de veículos em uma grande cidade, dividido em dois
subproblemas, a aquisição de dados de movimentação de veículos e a necessidade
do uso de técnicas sofisticadas para simplificar a visualização já que não é
viável extrair muitas informações simplesmente desenhando todos dados brutos e
também dos seus múltiplos atributos que esses dados podem conter, como direção,
velocidade, etc.  Afirmo ainda que visualizar essa grande quantidade de dados é
um grande desafio em termos de escalabilidade visual e computacional.

% A Proposta e justificativa
Proponho então para o primeiro problema o uso de dados simulados de um
simulador de cidades inteligentes, capaz de simular o tráfego de veículos de
uma grande cidade como São Paulo. Para o segundo problema proponho o uso do
bundling para visualização do dados de movimentação de veículos de
forma que ela possa auxiliar na exploração da dinâmica do trânsito na cidade e
consequentemente no planejamento da mobilidade urbana.

Referências para esse capítulo: \cite{Andrienko2011}, \cite{Liu2013},
\cite{Hurter2013}, \cite{Zhou2013}
