%% ------------------------------------------------------------------------- %%
\chapter{Introdução}
\label{cap:introducao}

% Contexto e motivação
Com a crescente disponibilidade de tecnologias para sistemas de localização,
como GPS, ocorre concomitantemente um grande aumento na geração de dados de
movimentação, que registram a localização de veículos, telefones celulares e
até mesmo animais, se estendendo por uma variedade de aplicações.

Focamos então no tráfego de veículos, buscando entender
o seu fluxo e sua dinâmica a partir de simulações que registram a movimentação
de veículos no dia a dia de uma grande cidade.  Esses dados, são medidas espaço
temporais e podem conter vários atributos como
a velocidade, direção ou outras informações pertinentes de um objeto em movimentação,
constituindo um rastro deste objeto. A captura, análise e visualização de tais
dados é um amplo e interessante campo de estudo. No contexto deste trabalho, utilizamos
o rastro de veículos para  entender a estrutura do tráfego na cidade, como um
estudo que que busca viabilizar ferramentas para um melhor planejamento do transporte urbano.

Sistemas para monitoramento do tráfego podem ajudar a detectar congestionamentos,
evitar acidentes e dar suporte para uma melhor tomada de decisões. Setores de tráfego
aéreo e marítimo se apoiam em sistemas para fazer o monitoramento de aviões
e embarcações, como mostram os trabalhos de Tijmein e Xablau do mar.

% Problema
Fazer uma análise da movimentação de veículos pode não ser uma tarefa simples dada
a grande quantidade de informação. Pensando em um serviço como o Uber, por exemplo,
um simples conjunto de dados de um dia pode facilmente obter milhares de trajetórias e milhões
de entradas, o que traz desafios interessantes para se atingir uma boa escalabilidade
visual e computacional. Desenhar todos os caminhos dos veículos em uma tela  não é uma solução
adequada já que a sobreposição de várias rotas irá naturalmente gerar uma confusão na visualização.
Há também o desafio computacional de se explorar uma grande massa de dados, principalmente
falando no tráfego de veículos de uma grande cidade como São Paulo, 
que segundo o órgão de fiscalização
Detran\footnote{www.detran.sp.gov.br}, tem mais de 8 milhões de veículos.

% A Proposta
Uma visualização iterativa com a aplicação de técnicas avançadas de visualização
pode auxiliar na identificação de padrões no trânsito e ajudar a responder questões como,
qual é a densidade e distribuição dos veículos pelas vias da cidade, quais vias
estão mais congestionadas, como os congestionamentos afetam o trânsito em
geral, quais vias concentram o maior fluxo de veículos, etc. Essas questões a
serem respondidas podem auxiliar na tomada de decisões e na gestão da mobilidade
urbana de forma que se obtenha uma maior eficiência do transporte
veicular e da qualidade de vida dos cidadãos.

\section{Objetivos}
\label{sec:objetivos}

O objetivo deste trabalho é identificar padrões de movimentação dos veículos que
se movem pelas vias da cidade ao longo do tempo com o uso de técnicas
de visualização avançadas, buscando uma resposta para as seguintes
questões de pesquisa:

\begin{enumerate}
  \item Como visualizar dados simulados do tráfego de uma grande cidade como São Paulo?

  \item Como identificar padrões de movimentação dos veículos que se movem pelas vias
  da cidade ao longo do tempo?
        
  \item Como as técnicas de visualização (bundling, density maps) podem auxiliar na
    identificação de padrões no trânsito?
\end{enumerate}

Para atingir o objetivo geral, os seguintes objetivos específicos devem ser atingidos:

\begin{itemize}
  \item Obtenção e tratamento dos dados simulados do tráfego de veículos na cidade de São Paulo
  \item Pesquisa de ferramentas e tecnologias para desenvolvimento da ferramenta de visualização
  \item Implementação da ferramenta de visualização dos dados 
  \item Análise e interpretação dos resultados
\end{itemize}

\section{Estrutura do Trabalho}

As seções tal tal descrevem . . .
