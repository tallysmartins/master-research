%% ------------------------------------------------------------------------- %%
\chapter{Introdução}
\label{cap:introducao}

% Contexto e motivação
Apresentação do contexto argumentando sobre o aumento da geração de dados de
movimentação (geolocalizados) e suas aplicações em diversas áreas como carros,
pessoas,etc. Em seguida é citado o foco do trabalho no tráfego de veículos a
partir de simulações bem como a motivação para analisar esses dados na
perspectiva de estudos que buscam melhorias para o transporte urbano, apontando
ainda outros estudos que usam dados de movimentação em outros contextos, como
no tráfego de embarcações e aviões.


% Problema
Encadeado no parágrafo anterior é apresentado o problema em se fazer
uma análise da movimentação de veículos dados os desafios de escalabilidade
visual e computacional e embasamos o problema apontando como é um cenário que
pode chegar a milhões de dados em uma grande cidade como São Paulo.

% A Proposta e justificativa
Por fim é introduzida a proposta do trabalho em se
desenvolver uma visualização interativa com a aplicação de técnicas de
visualização avançadas e como elas podem auxiliar na identificação de padrões
no trânsito, como congestionamentos, utilização das vias e dar suporte 
à gestão da mobilidade urbana.

\section{Objetivos}
\label{sec:objetivos}

O principal objetivo do trabalho é selecionar e implementar técnicas de
visualização para identificar padrões de movimentação dos veículos que se movem
pelas vias da cidade ao longo do tempo. Dado esse objetivo foram levantadas até
o momento três questões de pesquisa:

\begin{enumerate} 
  \item Como visualizar dados simulados do tráfego de uma grande cidade como
São Paulo?

  \item Como identificar padrões de movimentação dos veículos que se movem
pelas vias da cidade ao longo do tempo?
        
  \item Que técnicas de visualização podem auxiliar na identificação de padrões
no trânsito?
\end{enumerate}

%Para atingir o objetivo geral, os seguintes objetivos específicos devem ser atingidos:
%
%\begin{itemize}
%  \item Obtenção dos dados simulados do tráfego de veículos na cidade de São Paulo
%  \item Seleção de Técnicas de Visualização
%  \item Pesquisa de ferramentas e tecnologias para aplicação das técnicas
%  \item Implementação das técnicas de visualização sobre os dados da simulação
%  \item Análise e avaliação dos resultados
%\end{itemize}
