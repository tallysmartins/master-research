%% ------------------------------------------------------------------------- %%
\chapter{Introdução}
\label{cap:introducao}

% Contexto e motivação
Aqui eu darei uma introdução do contexto e da motivação, argumentando sobre o aumento da geração de dados de
movimentação (geolocalizados) e suas aplicações em diversas áreas como carros,
pessoas,etc. Em seguida irei deixar claro o foco do trabalho no tráfego de veículos a
partir de simulações, defendendo o porque de analisar esses dados na
perspectiva de estudos que buscam melhorias para o transporte urbano e como simulador
pode auxiliar nesse processo. Aponto ainda outros estudos que usam dados de
movimentação em outros contextos, como no tráfego de embarcações e aviões.

% Problema
Encadeado no parágrafo anterior eu apresento o problema em se fazer
uma análise da movimentação de veículos em uma grande cidade como São Paulo, que apresenta
desafios de escalabilidade visual e computacional, encaixando no contexto de visualização de dados
espaço temporais que são comumente representados por grafos dinâmicos. Abordo a relevância
do problema comentando a complexidade em se construir visualizações com pouca oclusão
e que codifiquem vários atributos, como velocidade, direção e que ainda que mostrem
correlações com outras fontes de dados, como dados climáticos ou socio-econômicos. 
Afirmo ainda  necessidade de que tais ferramentas e que elas podem auxiliar à tomada
de decisões e na busca de melhorias na gestão da mobilidade urbana.

% A Proposta e justificativa
Por fim eu apresento a proposta do trabalho em se
desenvolver uma ferramenta para visualização interativa com a aplicação de técnicas de
visualização que permitam enxergar uma grande quantidade de dados do trânsito
obtidos a partir de um simulador de cidades inteligentes. Defendo também que tal
ferramenta pode ainda ser utilizada de maneira a analisar quaisquer outras fontes
de dados de movimentação com características similares, como dados
reais de ônibus, táxi ou outros veículos que se movem nas ruas da cidades 
(futuros carros voadores da uber?)

\section{Objetivos}
\label{sec:objetivos}

O principal objetivo do trabalho é selecionar desenvolver uma visualização que
mostre o fluxo de veículos e vários atributos como, direção, velocidade e sua
dinâmica ao longo do tempo. Dado esse objetivo foram levantadas até
o momento três questões de pesquisa:

\begin{enumerate} 
  \item Como visualizar dados simulados do tráfego de uma grande cidade como
São Paulo?

  \item Como identificar padrões de movimentação dos veículos que se movem
pelas vias da cidade ao longo do tempo?
        
  \item Que técnicas de visualização podem auxiliar na identificação de padrões
no trânsito?
\end{enumerate}

%Para atingir o objetivo geral, os seguintes objetivos específicos devem ser atingidos:
%
%\begin{itemize}
%  \item Obtenção dos dados simulados do tráfego de veículos na cidade de São Paulo
%  \item Seleção de Técnicas de Visualização
%  \item Pesquisa de ferramentas e tecnologias para aplicação das técnicas
%  \item Implementação das técnicas de visualização sobre os dados da simulação
%  \item Análise e avaliação dos resultados
%\end{itemize}
