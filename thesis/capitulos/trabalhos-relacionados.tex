\chapter{Trabalhos Relacionados}
\label{cap:trabalhos-relacionados}

Nesse capítulo eu faço considerações sobre trabalhos relevantes na análise e
visualização de dados de movimentação e que utilizam bundling .  De forma
geral, tais trabalhos proporcionaram uma base para compor um conhecimento maior
sobre  a área. Pretendo falar sobre qual algoritmo de bundling usado e se houve
alguma peculiaridade, qual tamanho do dataset, que tipo de padrões o trabalho
apresentou na visualização

\begin{description}
  \item[Visualizing Interchange Patterns in Massive Movement Data:] esse
trabalho apresenta um diagrama circular que mostra a direção do fluxo entre
diferentes nós que simbolizam um ponto de interesse. Ele utiliza como estudo de
caso estações de metrô de Singapura com dados sigilosos do sistema de
transporte. \citep{Zeng2013} 

\item[Dynamic Multi Scale Visualization of Flight Data:]
 Esse trabalho do apresenta como visualizar uma grande quantidade
de dados do tráfego aéreo. \citep{Klein2014}

\item[Visualization of vessel movements:]
 Esse trabalho mostra a dinâmica do tráfego de embarcações. \citep{Willems2009}

\item[Untangling origin-destination flows in geographic information systems:]
 Esse trabalho mostra a implementação do bundling como plugin de uma ferramenta GIS.
O algoritmo apresentado ainda recebe uma modificação ao algoritmo de bundling FDEB. \citep{Anita2017}

\item[Divided Edge Bundling for Directional Network Data:]
  Esse trabalho mostra uma abordagem para diferenciar no bundling a direção do movimento.
\citep{Selassie2011}
\end{description}

%  Por razões críticas, a movimentação de aeronaves no tráfego aéreo é monitorada
%  com grande cuidado. Sistemas avançados de localização registram em tempo real
%  a posição de aeronaves que sobrevoam o planeta. Esses sistemas podem ainda
%  conter outras informações como direção, altitude, velocidade, temperatura, pressão
%  e diversos outros atributos ao longo do trajeto. O tráfego aéreo é de certa
%  forma similar à movimentação de veículos sobre as vias da cidade e visualizar
%  essas informações traz também grandes desafios.
%
%  cite{Alex}, mostra um trabalho onde ele descreve uma visualização que utiliza
%  técnicas baseadas em imagem para análise exploratória de uma grande massa
%  de dados do tráfego. Sua análise inclui a utilização de três técnicas, bundling,
%  density maps e animações. Ele argumenta que, com essas técnicas, é possível visualizar
%  uma grande quantidade de dados, tanto a posição instantânea dos aviões como também
%  sua dinâmica ao longo do tempo, sem uma grande oclusão da visualização. 
%  Dentre os resultados apresentados está a detecção de \textit{outliers}, padrões
%  e congestionamentos durante vôos.
%
%  
%  O seu sistema possui alguns parâmetros configuráveis que permitem uma exploração dos dados que
%  trazem vários insights para visualização de padrões e outliers nos dados.
%
%-> Highlights:
%  - exploração das informações disponíveis e detecção de outliers
%  - visão de grandes áreas do tráfego e durante longos períodos (e.g. dados do mundo durante 1 mês)
%  - adaptação de várias técnicas baseadas em imagem (bundling, animation, density maps) para visualizar
%  padrões ao longo do tempo
%  - visualização com pouca oclusão
%  - análise em "tempo real" com processamento na GPU
%  - real world datasets
%
%-> Limitações
%  - Ainda gera alguma oclusão, como no dataset do mundo todo
%  - Software não disponível
%  - não implementa queries para filtros (e.g. aviões de altitude maior que X)
%  - Suas trilhas apresentam apenas três atributos (velocidade, direção, altura)

