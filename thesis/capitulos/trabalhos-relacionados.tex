\chapter{Trabalhos Relacionados}
\label{cap:trabalhos-relacionados}

Nesse capítulo eu faço considerações sobre trabalhos relevantes na análise e
visualização de dados de movimentação.  De forma geral, tais trabalhos
proporcionaram uma base para compor um conhecimento maior sobre visualização de
dados geoespaciais.

\section{Trabalhos sobre Visualização de Dados de Movimentação}

Aqui eu falo sobre três trabalhos relacionados que eu considero importantes que utilizam dados de movimentação,
para encontrar padrões (não necessariamente de tráfego de veículos). O objetivo
é mostrar que esses trabalhos respondem perguntas diferentes com métodos
e visualizações diferentes pra dar uma ideia desse universo.

\subsection{Visualizing interchange patterns in massive movement data}
 Esse trabalho aborda uma maneira de se detectar padrões de mudança de direção
usando como caso de uso a movimentação de pessoas na cidade de Singapura.

\subsection{Revealing Patterns and Trends of Mass Mobility through Spatial
and Temporal Abstraction of Origin-Destination Movement Data}

\subsection{Visualização do Tráfego Aéreo}
 Esse trabalho do Alex Telea apresenta como visualizar uma grande quantidade
de dados do tráfego aéreo, encontrando padrões de congestionamento e mudanças
ao longo do tempo. Seu contexto se parece bastante com o do meu trabalho
e responde perguntas interessantes para o estudo do tráfego de veículos, sendo
um grande candidato por onde começar na análise dos dados do simulador.

\section{Trabalhos sobre bundling}

\subsection{Divided Edge Bundling for Directional Network Data}
  Um trabalho que utiliza bundling dividido para separar os dados pela direção
do movimento e também utiliza um algoritmo de clusterização k-means para agrupar
trajetórias com origem-destino similares antes de aplicar o bundling propriamente
dito. (esse trabalho é legal pois separar pela direção é importante)

\subsection{Bundled Visualization of Dynamic Graph and Trail Data}
  Um trabalho que traz novas técnicas para o bundling, mapeando o algoritmo
no que chamamos de image-based, que é o fato de se gerar uma imagem do grafo
e depois aplicar o bundling. Isso faz com que a técnica mapeie bem para implementações
na GPU. (esse trabalho é legal porque traz uma técnica mais escalável do bundling e como
aplicar em grafos stream)

%  Por razões críticas, a movimentação de aeronaves no tráfego aéreo é monitorada
%  com grande cuidado. Sistemas avançados de localização registram em tempo real
%  a posição de aeronaves que sobrevoam o planeta. Esses sistemas podem ainda
%  conter outras informações como direção, altitude, velocidade, temperatura, pressão
%  e diversos outros atributos ao longo do trajeto. O tráfego aéreo é de certa
%  forma similar à movimentação de veículos sobre as vias da cidade e visualizar
%  essas informações traz também grandes desafios.
%
%  cite{Alex}, mostra um trabalho onde ele descreve uma visualização que utiliza
%  técnicas baseadas em imagem para análise exploratória de uma grande massa
%  de dados do tráfego. Sua análise inclui a utilização de três técnicas, bundling,
%  density maps e animações. Ele argumenta que, com essas técnicas, é possível visualizar
%  uma grande quantidade de dados, tanto a posição instantânea dos aviões como também
%  sua dinâmica ao longo do tempo, sem uma grande oclusão da visualização. 
%  Dentre os resultados apresentados está a detecção de \textit{outliers}, padrões
%  e congestionamentos durante vôos.
%
%  
%  O seu sistema possui alguns parâmetros configuráveis que permitem uma exploração dos dados que
%  trazem vários insights para visualização de padrões e outliers nos dados.
%
%-> Highlights:
%  - exploração das informações disponíveis e detecção de outliers
%  - visão de grandes áreas do tráfego e durante longos períodos (e.g. dados do mundo durante 1 mês)
%  - adaptação de várias técnicas baseadas em imagem (bundling, animation, density maps) para visualizar
%  padrões ao longo do tempo
%  - visualização com pouca oclusão
%  - análise em "tempo real" com processamento na GPU
%  - real world datasets
%
%-> Limitações
%  - Ainda gera alguma oclusão, como no dataset do mundo todo
%  - Software não disponível
%  - não implementa queries para filtros (e.g. aviões de altitude maior que X)
%  - Suas trilhas apresentam apenas três atributos (velocidade, direção, altura)

