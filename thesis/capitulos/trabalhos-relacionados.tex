\chapter{Conceitos e Trabalhos Relacionados}
\label{cap:trabalhos-relacionados}

Este capítulo levanta considerações sobre alguns trabalhos interpretados como
relevantes na análise de dados de movimentação e também alguns sistemas existentes
para controle do tráfego de veículos. De forma geral, tais trabalhos proporcionaram uma base
para compor um conhecimento maior sobre visualização de dados do trânsito.

\section{Dados de Movimentação}

Aqui caracterizamos com mais detalhes dados de movimentação, seus atributos
de maneira geral. Falarei também como o simulador é empregado para a obtenção
dos dados que compõe o rastro dos veículos e outras fontes de obtenção de dados
reais de movimentação do tráfego de veículos (onibus, taxi, etc).

\section{Sistemas para Controle do Trânsito}

Aqui citamos algumas ferramentas para controle do trânsito (seja de simulação, sejam do governo, ou
sejam de empresas privadas como scipopulis, que tipo de soluções elas prestam e/ou
que problemas elas apresentam e seu impacto no planejamento do transporte urbano.

\section{Trabalhos sobre Visualização de Dados de Movimentação}

Aqui são levantados alguns trabalhos relacionados que utilizam dados de movimentação,
para encontrar padrões (não necessariamente de tráfego de veículos). O objetivo
é mostrar que esses trabalhos respondem perguntas diferentes com métodos
e visualizações diferentes pra dar uma ideia desse universo.

De alguma maneira será apontado que as técnicas apresentadas no trabalho do
Alex Telea sobre análise do tráfego aéreo serão utilizadas no nosso trabalho.

\subsection{Visualizing interchange patterns in massive movement data}
 Esse trabalho relacionado aborda uma maneira de se detectar padrões de direção
na movimentação de pessoas na cidade de Singapura.

\subsection{Mapping to Cells: A Simple Method to Extract Traffic Dynamics from Probe Vehicle Data}
 Esse outro trabalho relacionado apresenta como um veículo sonda pode oferecer
dados para a análise da dinâmica do trânsito, detecção de congestionamentos, etc.

\subsection{Visualização do Tráfego Aéreo}

 Esse trabalho do Alex Telea apresenta como visualizar uma grande quantidade
de dados do tráfego aéreo, encontrando padrões de congestionamento e mudanças
ao longo do tempo. Aplica basicamente três técnicas de visualização, bundling,
windowing e density maps. Seu contexto se parece bastante com o do meu trabalho
e responde perguntas interessantes para o estudo do tráfego de veículos, sendo
um grande candidato por onde começar na análise dos dados do simulador. 


%  Por razões críticas, a movimentação de aeronaves no tráfego aéreo é monitorada
%  com grande cuidado. Sistemas avançados de localização registram em tempo real
%  a posição de aeronaves que sobrevoam o planeta. Esses sistemas podem ainda
%  conter outras informações como direção, altitude, velocidade, temperatura, pressão
%  e diversos outros atributos ao longo do trajeto. O tráfego aéreo é de certa
%  forma similar à movimentação de veículos sobre as vias da cidade e visualizar
%  essas informações traz também grandes desafios.
%
%  cite{Alex}, mostra um trabalho onde ele descreve uma visualização que utiliza
%  técnicas baseadas em imagem para análise exploratória de uma grande massa
%  de dados do tráfego. Sua análise inclui a utilização de três técnicas, bundling,
%  density maps e animações. Ele argumenta que, com essas técnicas, é possível visualizar
%  uma grande quantidade de dados, tanto a posição instantânea dos aviões como também
%  sua dinâmica ao longo do tempo, sem uma grande oclusão da visualização. 
%  Dentre os resultados apresentados está a detecção de \textit{outliers}, padrões
%  e congestionamentos durante vôos.
%
%  
%  O seu sistema possui alguns parâmetros configuráveis que permitem uma exploração dos dados que
%  trazem vários insights para visualização de padrões e outliers nos dados.
%
%-> Highlights:
%  - exploração das informações disponíveis e detecção de outliers
%  - visão de grandes áreas do tráfego e durante longos períodos (e.g. dados do mundo durante 1 mês)
%  - adaptação de várias técnicas baseadas em imagem (bundling, animation, density maps) para visualizar
%  padrões ao longo do tempo
%  - visualização com pouca oclusão
%  - análise em "tempo real" com processamento na GPU
%  - real world datasets
%
%-> Limitações
%  - Ainda gera alguma oclusão, como no dataset do mundo todo
%  - Software não disponível
%  - não implementa queries para filtros (e.g. aviões de altitude maior que X)
%  - Suas trilhas apresentam apenas três atributos (velocidade, direção, altura)


\section{Conclusão}

Aqui refletimos a vasta gama de possibilidades e a necessidade de se compor
uma ferramenta para análise dos dados do tráfego do simulador.

