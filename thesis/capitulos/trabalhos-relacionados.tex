\chapter{Trabalhos Relacionados}
\label{cap:trabalhos-relacionados}

Este capítulo levanta considerações sobre alguns trabalhos interpretados como
relevantes na análise de dados de movimentação e também alguns sistemas existentes
para controle do tráfego de veículos. De forma geral, tais trabalhos proporcionaram uma base
para compor um conhecimento maior sobre visualização de dados do trânsito.

\section{Dados de Movimentação}
Como sabemos, existe uma variedade de informações geradas que grava a movimentação
de vários objetos como carros, navios, aeronaves e até animais. Neste trabalho
focamos em dados simulados.

\subsection{Sistemas para Controle do Trânsito}

Citar trabalhos e pesquisar algumas ferramentas que são utilizadas nesse sentido.
Falar com a scipopulis, etc... 

\subsection{Visualização do Tráfego de Veículos}

Dois outros trabalhos foram encontrados, eles abordam diferentes visualizações
para também encontrar padrões no trânsito. blá blá

\textbf{Visualizing interchange patterns in massive movement data}

\textbf{Mapping to Cells: A Simple Method to Extract Traffic Dynamics from Probe Vehicle Data}

\subsection{Visualização do Tráfego Aéreo}

  Por razões críticas, a movimentação de aeronaves no tráfego aéreo é monitorada
  com grande cuidado. Sistemas avançados de localização registram em tempo real
  a posição de aeronaves que sobrevoam o planeta. Esses sistemas podem ainda
  conter outras informações como direção, altitude, velocidade, temperatura, pressão
  e diversos outros atributos ao longo do trajeto. O tráfego aéreo é de certa
  forma similar à movimentação de veículos sobre as vias da cidade e visualizar
  essas informações traz também grandes desafios.

  cite{Alex}, mostra um trabalho onde ele descreve uma visualização que utiliza
  técnicas baseadas em imagem para análise exploratória de uma grande massa
  de dados do tráfego. Sua análise inclui a utilização de três técnicas, bundling,
  density maps e animações. Ele argumenta que, com essas técnicas, é possível visualizar
  uma grande quantidade de dados, tanto a posição instantânea dos aviões como também
  sua dinâmica ao longo do tempo, sem uma grande oclusão da visualização. 
  Dentre os resultados apresentados está a detecção de \textit{outliers}, padrões
  e congestionamentos durante vôos.

  
  O seu sistema possui alguns parâmetros configuráveis que permitem uma exploração dos dados que
  trazem vários insights para visualização de padrões e outliers nos dados.

-> Highlights:
  - exploração das informações disponíveis e detecção de outliers
  - visão de grandes áreas do tráfego e durante longos períodos (e.g. dados do mundo durante 1 mês)
  - adaptação de várias técnicas baseadas em imagem (bundling, animation, density maps) para visualizar
  padrões ao longo do tempo
  - visualização com pouca oclusão
  - análise em "tempo real" com processamento na GPU
  - real world datasets

-> Limitações
  - Ainda gera alguma oclusão, como no dataset do mundo todo
  - Software não disponível
  - não implementa queries para filtros (e.g. aviões de altitude maior que X)
  - Suas trilhas apresentam apenas três atributos (velocidade, direção, altura)

\subsection{Outras análises de movimentação}

\section{Conclusão}

