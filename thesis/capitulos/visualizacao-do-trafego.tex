\chapter{Referencial teórico}
\label{cap:referencial-teorico}

\section{Análise de trajetórias}

Aqui eu falo um pouco sobre visualização de trajetórias e sua relação com
visualização de grafos. Abordo também alguns métodos para esse propósito, como
flow maps e vector fields e referencio trabalhos que os utilizam. Abordo então
o bundling como opção para visualizar esses dados e reduzir a oclusão da visualização.
Após a introdução do bundling eu coloco uma seção para aprofundar nas características
da técnica, algoritmos, aplicação e outras discussões importantes.

\section{Bundling}
    Aqui eu dou a definição formal e detalho os aspectos da técnica, variações,
algoritmos, contextos de uso, etc.

\subsection{Bundling Estático}
    Aqui eu falo sobre bundling em contextos estáticos, como quando você quer
ver um conjunto específico de dados já completo.

\subsection{Bundling Dinâmico}
    Aqui eu falo sobre bundling onde os dados e o intervalo de tempo mudam
dinamicamente, nisso há um problema a resolver: Recalcula todo o bundling ou
agrega o novo dado ao bundling já calculado?

\subsection{Bundling no tráfego de veículos}
    Aqui é mais como eu vou usar a técnica no meu trabalho, as limitações que
eu vou lidar (como a distorção espacial e a visualização dos múltiplos modais)

\section{Simulação do Tráfego de Veículos}

\subsection{InterSCSimulator}
Aqui eu detalho um pouco mais sobre o simulador e como ele pode ser usado para
suportar o estudo do tráfego nas cidades, impacto de mudanças, etc. Falo também
sobre a pesquisa origem destino e brevemente sobre o modelo de simulação empregado
que gera a movimentação ponto a ponto dos veículos levando em consideração diversos
modais de transporte, etc.

