%% ------------------------------------------------------------------------- %%
\chapter{Introdução}
\label{cap:introducao}

A produção de dados de movimentação tem aumentado bastante nos ultimos anos
(veiculos, navios, aviões e até animais)...
Focamos em dados simulados do transporte...


%% ------------------------------------------------------------------------- %%
\section{Justificativa}
\label{sec:justificativa}

O tráfego de veículos é um importante fator no contexto da mobilidade urbana
e oferece um cenário interessante para a aplicação das técnicas de visualização
apresentadas por cite{7294794}. Visualizar os dados do tráfego de uma grande
cidade como São Paulo, que segundo o órgão de fiscalização Detran\footnote{www.detran.sp.gov.br},
tem mais de 8 milhões de veículos, apresenta grandes desafios para o desenvolvimento de uma solução que
tenha uma escalabilidade visual e computacional, o que reforça ainda mais a
relevância científica de seu trabalho.

Uma visualização iterativa com a aplicação de técnicas avançadas pode auxiliar
na identificação de padrões no trânsito e ajudar a responder questões como,
qual é a densidade e distibuição dos veículos pelas vias da cidade, quais vias
estão mais congestionadas, como os congestionamentos afetam o trânsito em geral,
quais vias concentram o maior fluxo de veículos, etc. Essas questões
a serem respondidas talvez auxiliem na tomada de decisões sobre o planejamento e
gestão das cidade de forma que se obtenha uma maior eficiência do transporte
veicular e da qualidade de vida dos cidadãos.

E ainda, a proposta deste trabalho faz uma ponte entre dois diferentes estudos,
o InterSCity Simulator, um simulador para cenários de cidades inteligentes capaz
de simular dados em larga escala, apresentado por cite{mabs2017}, e o estudo de visualização
de dados espaço-temporais que identifica padrões de movimentação no tráfego
aéreo, apresentado por cite{7294794}.

\section{Objetivos}
\label{sec:objetivos}

O objetivo deste trabalho é identificar padrões de movimentação dos veículos que
se movem pelas vias da cidade ao longo do tempo com o uso de técnicas
de visualização avançadas, buscando uma resposta para as seguintes
questões de pesquisa:

\begin{enumerate}
  \item Como visualizar dados simulados do tráfego de uma grande cidade como São Paulo?

  \item Como identificar padrões de movimentação dos veículos que se movem pelas vias
  da cidade ao longo do tempo?
        
  \item Como as técnicas de visualização (bundling, density maps) podem auxiliar na
    identificação de padrões no trânsito?
\end{enumerate}

Para atingir o objetivo geral, os seguintes objetivos específicos devem ser atingidos:

\begin{itemize}
  \item Obtenção e tratamento dos dados simulados do tráfego de veículos na cidade de São Paulo
  \item Pesquisa de ferramentas e tecnologias para desenvolvimento da ferramenta de visualização
  \item Implementação da ferramenta de visualização dos dados 
  \item Análise e interpretação dos resultados
\end{itemize}

\section{Estrutura do Trabalho}

As seções tal tal descrevem . . .
